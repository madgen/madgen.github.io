---
title: Concepts in Programming Languages II
date: 2018-04-29
---

\begin{document}

\begin{enumerate}
  \item Do type inference (by hand or typesetter) for the following ML function:

    \begin{minted}{ml}
fn f => fn g => fn x => f (g (f x))
    \end{minted}

    This involves providing a derivation tree as well as the constraint
    generated and their unification.

  \item This question is about ML module system.

    \begin{enumerate}[(a)]
      \item Provide two implementations using SML module system for the
        following signature for stack abstract data type:

        \begin{minted}{ml}
signature STACK =
  sig
    type 'a t
    exception E;

    val empty : 'a t
    val push  : ('a t * 'a) -> 'a t
    val pop   : 'a t -> 'a t
    val top   : 'a t -> 'a
  end
      \end{minted}

        The first structure implementing the signature should use a list for
        its internal representation and the second one should use a new data
        type with constructors \texttt{push} and \texttt{empty}.

      \item With a separate structure assignment create an abstract data type
        out of the second stack structure. This one must implement the
        signature opaquely.  In comments explain what this means and what
        operation(s) is prohibited compared to transparent implementation.

      \item Define a functor that takes a STACK and generates an EVALUATOR as
        defined below. Your functor should generate a reverse polish adder.

        \begin{minted}{ml}
datatype StackElement = OpPlus | OpInt of int;

signature EVALUATOR =
  sig
    type t
    val empty   : t
    val push    : (t * StackElement) -> t
    val top     : t -> int
  end
        \end{minted}
    \end{enumerate}

  \item What is the difference between parallelism and concurrency? Under which
    circumstances each of them improve performance benefits?

    How a pthreads different from Erlang processes?

  \item How are recent (last 15 years) advances in hardware have blurred the
    distinction between distributed systems and single machine parallelism?
    Give example technologies/language constructs from the course that
    illustrate your answer.

  \item Please read~\citet{wadler1995monads} up to and including Section 3.
    This is the paper I found most useful when I was first learning about
    monads.

    If you don't find this paper helpful, google ``Monad tutorial'' and
    you'll be presented with hundreds of blog posts each often employs some
    form of an analogy to explain. Be warned, they might misrepresented the
    concept, or confuse you due to over simplification.

  \item As many tripos questions as you’d like me to mark.

\end{enumerate}

\end{document}
