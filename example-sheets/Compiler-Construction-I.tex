---
title: Compiler Construction Example Sheet I
date: 2018-01-06
---

\begin{document}

\begin{enumerate}
  \item Build and play with the SLANG compiler. Experiment with the front end
    (syntax/type checker) and be prepared to discuss what you have done in the
    supervision. It is OK if the changes you make do not work.

  \item Write down a \emph{pipeline} of steps involved in compilation paying
    special attention to what goes into each step and what goes out. You can be
    as vague or specific as you want, but be prepared to discuss compiler
    related concepts and how it fits to your pipeline.

  \item How is \textsc{LLVM} approach to compiler construction is different
    than that is presented in this course?

  \item Why are functional languages are commonly used to prototype compilers?

  \item Why does \texttt{past.ml} in first Slang compiler have \texttt{loc}
    parameter for every constructor, while the corresponding constructors in
    \texttt{ast.ml} don't have them?

  \item According to lexical analysis algorithm described in slide 31, what are
    the two rules that disambiguate multiple possible matches out of the same
    character stream?

  \item Starting with regular expressions that match individual tokens, how do
    we generate a single lexing program?

  \item What is the difference between concrete and abstract syntax trees (CST
    vs AST)? What stages of compilation involve these?

  \item What does it mean for a grammar to be ambiguous? How can this ambiguity
    be resolved?

  \item How does a recursive descent parser work? What class of languages can it
    be used to describe? What are the problems with it?

  \item How does a shift-reduce parser work? What class of languages can it be
    used to describe? What problems of recursive decent parser does it address?

  \item \href{http://www.cl.cam.ac.uk/teaching/exams/pastpapers/y2012p3q4.pdf}{2012/3/4} (a) (b)

  \item \href{http://www.cl.cam.ac.uk/teaching/exams/pastpapers/y2015p3q3.pdf}{2015/3/3}
\end{enumerate}

Thanks to Zébulon Goriely for catching a typo.

\end{document}
