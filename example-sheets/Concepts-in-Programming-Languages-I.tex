---
title: Concepts in Programming Languages I
date: 2018-04-29
---

\begin{document}

\begin{enumerate}
  \item Read and make notes on chapters 3 (on \textsc{LISP}) \& 5 (on
    \textsc{ALGOL}) of \citet{mitchell2003concepts}. In your notes, pay
    particular attention to motivation and new innovations of the language.

  \item An author writes

    \begin{displayquote}
      Most successful language design efforts share three important
      characteristics:

      \begin{itemize}
        \item Motivating Application: The language was designed so that a
          specific kind of program could be written more easily,

        \item Abstract Machine: There is a simple and unambiguous program
          execution model, and

        \item Theoretical Foundations: Theoretical understanding was the basis
          for including certain capabilities and omitting others.
      \end{itemize}
    \end{displayquote}

    Briefly discuss the merits and/or shortcomings of the above three
    statements, giving examples and/or counterexamples from procedural,
    applicative, logical, and/or object-oriented programming languages.

  \item For the programming languages \textsc{FORTRAN}, \textsc{LISP},
    \textsc{Java}, \textsc{C}, \textsc{C++} and \textsc{ML} briefly discuss and
    evaluate their typing disciplines.

  \item Consider the following two program fragments:
    \begin{minted}{lisp}
( defvar x 1 )
( defun g(z) (+ x z) )
( defun f(y)
 (+ (g 1)
    ( let
       ( ( x (+ y 3) ) )
       (
           g(+ y x)
           ) ) ) )
( f 2 )
    \end{minted}

    \begin{minted}{ml}
val x = 1 ;
fun g(z) = x + z ;
fun f(y)
= g(1) +
  let
     val x = y + 3
  in
     g(y+x)
  end ;
f(2) ;
    \end{minted}

    What are their respective output values when run in their corresponding
    interpreters? Justify your answer, relating it to the concepts in the
    course.

  \item Give an overview of the LISP abstract machine (or execution model) and
    comment on its merits and drawbacks from the viewpoints of programming,
    compilation, execution, etc.

  \item Define the following parameter-passing mechanisms:
    pass-by-value,pass-by-reference, pass-by value/result, and pass-by-name.
    Briefly comment on their merits and drawbacks.

  \item What is aliasing in the context of programming languages? Explain the
    contexts in which it arises and provide examples of the phenomenon.

  \item As many exam questions as you would like me to mark.

\end{enumerate}

\end{document}
