---
title: Compiler Construction Example Sheet II
date: 2023-02-13
---

\begin{document}

\begin{enumerate}
  \item Read the paper by~\citet{reynolds1972definitional} up to and including
    section seven. Later sections are delightful as well, but less relevant to
    the course. Section one can be skimmed, but it nicely sets up the scene. He
    talks about \textsc{Algol} a lot. For the unfamiliar it is the archetype of
    block structured programming languages \emph{e.g.} \textsc{C}, but unlike
    \textsc{C} it has \emph{some} high-order function support.

    This is a great paper in computer science. Apart from historical value, it
    is also what the lecturer basis his interpreter upon. It will allow you to
    deeply understand how and why we bother with defunctionalisation \&
    continuation passing style.

    I strongly suggest you read it before you attempt the rest of the workset.

  \item
    \href{https://www.cl.cam.ac.uk/teaching/2122/CompConstr/exercises_cps.ml}{CPS
    and defunctionalisation exercise sheet} (from the course site)

  \item \href{https://www.cl.cam.ac.uk/teaching/exams/pastpapers/y2020p4q4.pdf}{2020/4/4}

  \item \href{https://www.cl.cam.ac.uk/teaching/exams/pastpapers/y2022p4q2.pdf}{2022/4/2}

  \item \href{http://www.cl.cam.ac.uk/teaching/exams/pastpapers/y2015p3q3.pdf}{2015/3/3}
\end{enumerate}

\end{document}
