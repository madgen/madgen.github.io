---
title: Compiler Construction Example Sheet III
date: 2018-01-06
lastUpdated: 2018-02-15
---

\begin{document}

\begin{enumerate}
  \item Read the paper by~\citet{reynolds1972definitional} up to and including
    section seven. Later sections are delightful as well, but less relevant to
    the course. Section one can be skimmed, but it nicely sets up the scene. He
    talks about \textsc{Algol} a lot. For the unfamiliar it is the archetype of
    block structured programming languages \emph{e.g.} \textsc{C}, but unlike
    \textsc{C} it has \emph{some} high-order function support.

    This is a great paper in computer science. Apart from historical value, it
    is also what the lecturer basis his interpreter upon. It will allow you to
    deeply understand how and why we bother with defunctionalisation \&
    continuation passing style.

    I strongly suggest you read it before you attempt the rest of the workset.

  \item
    \href{https://www.cl.cam.ac.uk/teaching/current/CompConstr/exercises_cps.ml}{CPS
    and defunctionalisation exercise sheet} (from the course site)

  \item How is the state data type in interpreter 1 used? What is the
    difference between EXAMINE and COMPUTE constructors? Discuss in detail.

  \item How can multiple source files can be compiled separately and linked
    together? How and which parts of ELF help you do that?

  \item \href{http://www.cl.cam.ac.uk/teaching/exams/pastpapers/y2017p23q4.pdf}{2017/3/4}

  \item Write a program with the following specification on Linux outputting an
    ELF binary using x86\_64 instruction set:

    Input: An integer N (supplied as command line argument or standard input)

    Output: N lines. In the first of which there is a \texttt{*} and in each
    subsequent lines there is an extra star.

    Example:
    \begin{minted}{text}
Input: 5
Output:
*
**
***
****
*****
    \end{minted}

\end{enumerate}

\end{document}
