---
title: Compiler Construction Example Sheet III
date: 2018-01-06
lastUpdated: 2023-02-20
---

\begin{document}

\begin{enumerate}
  \item In \textsc{C}, when you are allocating memory with \texttt{malloc} you
    need to pass the size of the memory you want to be allocated and the
    routine returns a pointer. When you later want to free this memory you use
    the free routine and only pass the pointer. How come \emph{libc} knows how
    much memory to deallocate?

  \item How can multiple source files can be compiled separately and linked
    together? How and which parts of ELF help you do that?

  \item After reviewing lecture notes on garbage collection, read
    \citet{unified} up to and including section 4.

  \item How is environment implemented in Slang interpreters 0 and 2. Pay
    special attention to handling of functions.

  \item What is the main difference between interpreter 0 and interpreter 2?
    What are the purposes of this change? What are the additions to the
    interpreter 2 in the version 3? How do they improve the compiler?

  \item \href{https://www.cl.cam.ac.uk/teaching/exams/pastpapers/y2020p4q3.pdf}{2020/4/3}

  \item \href{https://www.cl.cam.ac.uk/teaching/exams/pastpapers/y2018p4q4.pdf}{2018/4/4}

  \item As you know Java virtual machine is a bytecode interpreter. Here are
    the instructions of a function \texttt{f}. What is the high-level code that
    produces these instructions? Annotate fragments of the code. This will
    require you to lookup what these instructions do online. Please do not use
    a decompiler.

    \begin{minted}{text}
public static int f(int);
Code:
   0: iload_0
   1: ifne          6
   4: iconst_1
   5: ireturn
   6: iload_0
   7: iload_0
   8: iconst_1
   9: isub
  10: invokestatic  #4     // Method f
  13: imul
  14: ireturn
    \end{minted}

  \item Write a program with the following specification on Linux outputting an
    ELF binary using x86\_64 instruction set:

    Input: An integer N (supplied as command line argument or standard input)

    Output: N lines. In the first of which there is a \texttt{*} and in each
    subsequent lines there is an extra star.

    Example:
    \begin{minted}{text}
Input: 5
Output:
*
**
***
****
*****
    \end{minted}

\end{enumerate}

\end{document}
